\documentclass[12pt, letterpaper]{article}
\usepackage[utf8]{inputenc}
\title{Chapter 3 Notes}
\date{June 15, 2022}

\begin{document}
    \textnormal{Haskell uses types and classes.} \newline
    \textnormal{A \textit{type} is a collection of related values. 
    For example, Bool is a type which can hold False and True. The correct notation for this is}
    \begin{math}
        v :: T
    \end{math}
    \textnormal{to mean that v is a value in the type T. 
    Every expression must have a type, which is calculated at runtime by \textit{type inference.}
    The type of a function is determined by its inputs and outputs. For example, we would say}\newline
    \begin{math}
        (+) :: Num \space a => a -> a -> a
    \end{math}\newline
    \textnormal{
        because plus takes two Num inputs a and outputs a num a as well. 
    }
\end{document}